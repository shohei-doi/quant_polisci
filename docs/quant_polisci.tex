\PassOptionsToPackage{unicode=true}{hyperref} % options for packages loaded elsewhere
\PassOptionsToPackage{hyphens}{url}
%
\documentclass[]{bxjsarticle}
\usepackage{lmodern}
\usepackage{amssymb,amsmath}
\usepackage{ifxetex,ifluatex}
\usepackage{fixltx2e} % provides \textsubscript
\ifnum 0\ifxetex 1\fi\ifluatex 1\fi=0 % if pdftex
  \usepackage[T1]{fontenc}
  \usepackage[utf8]{inputenc}
  \usepackage{textcomp} % provides euro and other symbols
\else % if luatex or xelatex
  \usepackage{unicode-math}
  \defaultfontfeatures{Ligatures=TeX,Scale=MatchLowercase}
\fi
% use upquote if available, for straight quotes in verbatim environments
\IfFileExists{upquote.sty}{\usepackage{upquote}}{}
% use microtype if available
\IfFileExists{microtype.sty}{%
\usepackage[]{microtype}
\UseMicrotypeSet[protrusion]{basicmath} % disable protrusion for tt fonts
}{}
\IfFileExists{parskip.sty}{%
\usepackage{parskip}
}{% else
\setlength{\parindent}{0pt}
\setlength{\parskip}{6pt plus 2pt minus 1pt}
}
\usepackage{hyperref}
\hypersetup{
            pdftitle={Rで計量政治学入門},
            pdfauthor={土井 翔平},
            pdfborder={0 0 0},
            breaklinks=true}
\urlstyle{same}  % don't use monospace font for urls
\usepackage{color}
\usepackage{fancyvrb}
\newcommand{\VerbBar}{|}
\newcommand{\VERB}{\Verb[commandchars=\\\{\}]}
\DefineVerbatimEnvironment{Highlighting}{Verbatim}{commandchars=\\\{\}}
% Add ',fontsize=\small' for more characters per line
\usepackage{framed}
\definecolor{shadecolor}{RGB}{248,248,248}
\newenvironment{Shaded}{\begin{snugshade}}{\end{snugshade}}
\newcommand{\AlertTok}[1]{\textcolor[rgb]{0.94,0.16,0.16}{#1}}
\newcommand{\AnnotationTok}[1]{\textcolor[rgb]{0.56,0.35,0.01}{\textbf{\textit{#1}}}}
\newcommand{\AttributeTok}[1]{\textcolor[rgb]{0.77,0.63,0.00}{#1}}
\newcommand{\BaseNTok}[1]{\textcolor[rgb]{0.00,0.00,0.81}{#1}}
\newcommand{\BuiltInTok}[1]{#1}
\newcommand{\CharTok}[1]{\textcolor[rgb]{0.31,0.60,0.02}{#1}}
\newcommand{\CommentTok}[1]{\textcolor[rgb]{0.56,0.35,0.01}{\textit{#1}}}
\newcommand{\CommentVarTok}[1]{\textcolor[rgb]{0.56,0.35,0.01}{\textbf{\textit{#1}}}}
\newcommand{\ConstantTok}[1]{\textcolor[rgb]{0.00,0.00,0.00}{#1}}
\newcommand{\ControlFlowTok}[1]{\textcolor[rgb]{0.13,0.29,0.53}{\textbf{#1}}}
\newcommand{\DataTypeTok}[1]{\textcolor[rgb]{0.13,0.29,0.53}{#1}}
\newcommand{\DecValTok}[1]{\textcolor[rgb]{0.00,0.00,0.81}{#1}}
\newcommand{\DocumentationTok}[1]{\textcolor[rgb]{0.56,0.35,0.01}{\textbf{\textit{#1}}}}
\newcommand{\ErrorTok}[1]{\textcolor[rgb]{0.64,0.00,0.00}{\textbf{#1}}}
\newcommand{\ExtensionTok}[1]{#1}
\newcommand{\FloatTok}[1]{\textcolor[rgb]{0.00,0.00,0.81}{#1}}
\newcommand{\FunctionTok}[1]{\textcolor[rgb]{0.00,0.00,0.00}{#1}}
\newcommand{\ImportTok}[1]{#1}
\newcommand{\InformationTok}[1]{\textcolor[rgb]{0.56,0.35,0.01}{\textbf{\textit{#1}}}}
\newcommand{\KeywordTok}[1]{\textcolor[rgb]{0.13,0.29,0.53}{\textbf{#1}}}
\newcommand{\NormalTok}[1]{#1}
\newcommand{\OperatorTok}[1]{\textcolor[rgb]{0.81,0.36,0.00}{\textbf{#1}}}
\newcommand{\OtherTok}[1]{\textcolor[rgb]{0.56,0.35,0.01}{#1}}
\newcommand{\PreprocessorTok}[1]{\textcolor[rgb]{0.56,0.35,0.01}{\textit{#1}}}
\newcommand{\RegionMarkerTok}[1]{#1}
\newcommand{\SpecialCharTok}[1]{\textcolor[rgb]{0.00,0.00,0.00}{#1}}
\newcommand{\SpecialStringTok}[1]{\textcolor[rgb]{0.31,0.60,0.02}{#1}}
\newcommand{\StringTok}[1]{\textcolor[rgb]{0.31,0.60,0.02}{#1}}
\newcommand{\VariableTok}[1]{\textcolor[rgb]{0.00,0.00,0.00}{#1}}
\newcommand{\VerbatimStringTok}[1]{\textcolor[rgb]{0.31,0.60,0.02}{#1}}
\newcommand{\WarningTok}[1]{\textcolor[rgb]{0.56,0.35,0.01}{\textbf{\textit{#1}}}}
\usepackage{longtable,booktabs}
% Fix footnotes in tables (requires footnote package)
\IfFileExists{footnote.sty}{\usepackage{footnote}\makesavenoteenv{longtable}}{}
\usepackage{graphicx,grffile}
\makeatletter
\def\maxwidth{\ifdim\Gin@nat@width>\linewidth\linewidth\else\Gin@nat@width\fi}
\def\maxheight{\ifdim\Gin@nat@height>\textheight\textheight\else\Gin@nat@height\fi}
\makeatother
% Scale images if necessary, so that they will not overflow the page
% margins by default, and it is still possible to overwrite the defaults
% using explicit options in \includegraphics[width, height, ...]{}
\setkeys{Gin}{width=\maxwidth,height=\maxheight,keepaspectratio}
\setlength{\emergencystretch}{3em}  % prevent overfull lines
\providecommand{\tightlist}{%
  \setlength{\itemsep}{0pt}\setlength{\parskip}{0pt}}
\setcounter{secnumdepth}{5}
% Redefines (sub)paragraphs to behave more like sections
\ifx\paragraph\undefined\else
\let\oldparagraph\paragraph
\renewcommand{\paragraph}[1]{\oldparagraph{#1}\mbox{}}
\fi
\ifx\subparagraph\undefined\else
\let\oldsubparagraph\subparagraph
\renewcommand{\subparagraph}[1]{\oldsubparagraph{#1}\mbox{}}
\fi

% set default figure placement to htbp
\makeatletter
\def\fps@figure{htbp}
\makeatother

\usepackage{booktabs}
\usepackage{amsthm}
\makeatletter
\def\thm@space@setup{%
  \thm@preskip=8pt plus 2pt minus 4pt
  \thm@postskip=\thm@preskip
}
\makeatother

\usepackage{zxjatype}
\setmainfont[BoldFont = Noto Serif CJK JP]{Noto Serif CJK JP Light}
\setsansfont[BoldFont = Noto Sans CJK JP]{Noto Sans CJK JP Light}
\setmonofont{Noto Sans Mono CJK JP}
\setCJKmainfont[BoldFont = Noto Sans CJK JP]{Noto Sans CJK JP Light}
\setCJKsansfont[BoldFont = Noto Sans CJK JP]{Noto Sans CJK JP Light}

\usepackage{xcolor}
\definecolor{main}{HTML}{53727d}

\usepackage{hyperref}
\hypersetup{
  colorlinks = true,
  allcolors = main,
  pdfauthor = {},
  pdftitle = {},
  pdfsubject = {},
  pdfkeywords = {}
}
\usepackage[]{natbib}
\bibliographystyle{apalike}

\title{Rで計量政治学入門}
\author{土井 翔平}
\date{2020-04-18}

\begin{document}
\maketitle

{
\setcounter{tocdepth}{2}
\tableofcontents
}
\hypertarget{index}{%
\section*{はじめに}\label{index}}
\addcontentsline{toc}{section}{はじめに}

本書はRによる計量政治学の入門レベルの講義資料です。
質問や間違いなどがありましたら、\href{mailto:shohei.doi0504@gmail.com}{ご連絡}を下さい。
筆者のプロフィールは\href{https://shohei-doi.github.io/}{こちら}をご覧ください。

RやRStudioのインストールについては\protect\hyperlink{install-r}{Rの環境構築}を、基本操作については\protect\hyperlink{intro-r}{プログラミング入門}をご覧ください。

\hypertarget{part-ux30c7ux30fcux30bfux30cfux30f3ux30c9ux30eaux30f3ux30b0}{%
\part{データ・ハンドリング}\label{part-ux30c7ux30fcux30bfux30cfux30f3ux30c9ux30eaux30f3ux30b0}}

\hypertarget{data-import}{%
\section{データの読み込み}\label{data-import}}

本章ではデータを読み込む方法について解説します。

\hypertarget{ux4f7fux7528ux3059ux308bux30d1ux30c3ux30b1ux30fcux30b8}{%
\subsection{使用するパッケージ}\label{ux4f7fux7528ux3059ux308bux30d1ux30c3ux30b1ux30fcux30b8}}

\begin{Shaded}
\begin{Highlighting}[]
\KeywordTok{library}\NormalTok{(tidyverse)}
\end{Highlighting}
\end{Shaded}

\hypertarget{appendix-ux88dcux8ad6}{%
\appendix}


\hypertarget{install-r}{%
\section{Rの環境構築}\label{install-r}}

\hypertarget{intro-r}{%
\section{プログラミング入門}\label{intro-r}}

\hypertarget{ux95a2ux6570}{%
\subsection{関数}\label{ux95a2ux6570}}

\textbf{関数}(function)とは何かを入力すると、何かを出力するものです。

\begin{itemize}
\tightlist
\item
  入力するものを入力引数(input argument)、出力するものを出力引数(output argument)と読んだりします。
\end{itemize}

\hypertarget{environment}{%
\section*{動作環境}\label{environment}}
\addcontentsline{toc}{section}{動作環境}

\begin{Shaded}
\begin{Highlighting}[]
\KeywordTok{sessionInfo}\NormalTok{()}
\end{Highlighting}
\end{Shaded}

\begin{verbatim}
## R version 3.6.3 (2020-02-29)
## Platform: x86_64-pc-linux-gnu (64-bit)
## Running under: Ubuntu 18.04.4 LTS
## 
## Matrix products: default
## BLAS:   /usr/lib/x86_64-linux-gnu/blas/libblas.so.3.7.1
## LAPACK: /usr/lib/x86_64-linux-gnu/lapack/liblapack.so.3.7.1
## 
## locale:
##  [1] LC_CTYPE=en_US.UTF-8       LC_NUMERIC=C              
##  [3] LC_TIME=en_US.UTF-8        LC_COLLATE=en_US.UTF-8    
##  [5] LC_MONETARY=en_US.UTF-8    LC_MESSAGES=en_US.UTF-8   
##  [7] LC_PAPER=en_US.UTF-8       LC_NAME=C                 
##  [9] LC_ADDRESS=C               LC_TELEPHONE=C            
## [11] LC_MEASUREMENT=en_US.UTF-8 LC_IDENTIFICATION=C       
## 
## attached base packages:
## [1] stats     graphics  grDevices utils     datasets  methods   base     
## 
## other attached packages:
## [1] forcats_0.5.0   stringr_1.4.0   dplyr_0.8.5     purrr_0.3.3    
## [5] readr_1.3.1     tidyr_1.0.2     tibble_3.0.0    ggplot2_3.3.0  
## [9] tidyverse_1.3.0
## 
## loaded via a namespace (and not attached):
##  [1] tidyselect_1.0.0 xfun_0.13        haven_2.2.0      lattice_0.20-41 
##  [5] colorspace_1.4-1 vctrs_0.2.4      generics_0.0.2   htmltools_0.4.0 
##  [9] yaml_2.2.1       rlang_0.4.5      pillar_1.4.3     withr_2.1.2     
## [13] glue_1.4.0       DBI_1.1.0        dbplyr_1.4.2     modelr_0.1.6    
## [17] readxl_1.3.1     lifecycle_0.2.0  munsell_0.5.0    gtable_0.3.0    
## [21] cellranger_1.1.0 rvest_0.3.5      evaluate_0.14    knitr_1.28      
## [25] fansi_0.4.1      broom_0.5.5      Rcpp_1.0.4.6     backports_1.1.6 
## [29] scales_1.1.0     jsonlite_1.6.1   fs_1.4.1         hms_0.5.3       
## [33] digest_0.6.25    stringi_1.4.6    bookdown_0.18    grid_3.6.3      
## [37] cli_2.0.2        tools_3.6.3      magrittr_1.5     crayon_1.3.4    
## [41] pkgconfig_2.0.3  ellipsis_0.3.0   xml2_1.3.1       reprex_0.3.0    
## [45] lubridate_1.7.8  assertthat_0.2.1 rmarkdown_2.1    httr_1.4.1      
## [49] rstudioapi_0.11  R6_2.4.1         nlme_3.1-144     compiler_3.6.3
\end{verbatim}

\bibliography{book.bib,packages.bib}

\end{document}
