\PassOptionsToPackage{unicode=true}{hyperref} % options for packages loaded elsewhere
\PassOptionsToPackage{hyphens}{url}
%
\documentclass[]{bxjsarticle}
\usepackage{lmodern}
\usepackage{amssymb,amsmath}
\usepackage{ifxetex,ifluatex}
\usepackage{fixltx2e} % provides \textsubscript
\ifnum 0\ifxetex 1\fi\ifluatex 1\fi=0 % if pdftex
  \usepackage[T1]{fontenc}
  \usepackage[utf8]{inputenc}
  \usepackage{textcomp} % provides euro and other symbols
\else % if luatex or xelatex
  \usepackage{unicode-math}
  \defaultfontfeatures{Ligatures=TeX,Scale=MatchLowercase}
\fi
% use upquote if available, for straight quotes in verbatim environments
\IfFileExists{upquote.sty}{\usepackage{upquote}}{}
% use microtype if available
\IfFileExists{microtype.sty}{%
\usepackage[]{microtype}
\UseMicrotypeSet[protrusion]{basicmath} % disable protrusion for tt fonts
}{}
\IfFileExists{parskip.sty}{%
\usepackage{parskip}
}{% else
\setlength{\parindent}{0pt}
\setlength{\parskip}{6pt plus 2pt minus 1pt}
}
\usepackage{hyperref}
\hypersetup{
            pdftitle={Rで計量政治学入門},
            pdfauthor={土井 翔平},
            pdfborder={0 0 0},
            breaklinks=true}
\urlstyle{same}  % don't use monospace font for urls
\usepackage{color}
\usepackage{fancyvrb}
\newcommand{\VerbBar}{|}
\newcommand{\VERB}{\Verb[commandchars=\\\{\}]}
\DefineVerbatimEnvironment{Highlighting}{Verbatim}{commandchars=\\\{\}}
% Add ',fontsize=\small' for more characters per line
\usepackage{framed}
\definecolor{shadecolor}{RGB}{248,248,248}
\newenvironment{Shaded}{\begin{snugshade}}{\end{snugshade}}
\newcommand{\AlertTok}[1]{\textcolor[rgb]{0.94,0.16,0.16}{#1}}
\newcommand{\AnnotationTok}[1]{\textcolor[rgb]{0.56,0.35,0.01}{\textbf{\textit{#1}}}}
\newcommand{\AttributeTok}[1]{\textcolor[rgb]{0.77,0.63,0.00}{#1}}
\newcommand{\BaseNTok}[1]{\textcolor[rgb]{0.00,0.00,0.81}{#1}}
\newcommand{\BuiltInTok}[1]{#1}
\newcommand{\CharTok}[1]{\textcolor[rgb]{0.31,0.60,0.02}{#1}}
\newcommand{\CommentTok}[1]{\textcolor[rgb]{0.56,0.35,0.01}{\textit{#1}}}
\newcommand{\CommentVarTok}[1]{\textcolor[rgb]{0.56,0.35,0.01}{\textbf{\textit{#1}}}}
\newcommand{\ConstantTok}[1]{\textcolor[rgb]{0.00,0.00,0.00}{#1}}
\newcommand{\ControlFlowTok}[1]{\textcolor[rgb]{0.13,0.29,0.53}{\textbf{#1}}}
\newcommand{\DataTypeTok}[1]{\textcolor[rgb]{0.13,0.29,0.53}{#1}}
\newcommand{\DecValTok}[1]{\textcolor[rgb]{0.00,0.00,0.81}{#1}}
\newcommand{\DocumentationTok}[1]{\textcolor[rgb]{0.56,0.35,0.01}{\textbf{\textit{#1}}}}
\newcommand{\ErrorTok}[1]{\textcolor[rgb]{0.64,0.00,0.00}{\textbf{#1}}}
\newcommand{\ExtensionTok}[1]{#1}
\newcommand{\FloatTok}[1]{\textcolor[rgb]{0.00,0.00,0.81}{#1}}
\newcommand{\FunctionTok}[1]{\textcolor[rgb]{0.00,0.00,0.00}{#1}}
\newcommand{\ImportTok}[1]{#1}
\newcommand{\InformationTok}[1]{\textcolor[rgb]{0.56,0.35,0.01}{\textbf{\textit{#1}}}}
\newcommand{\KeywordTok}[1]{\textcolor[rgb]{0.13,0.29,0.53}{\textbf{#1}}}
\newcommand{\NormalTok}[1]{#1}
\newcommand{\OperatorTok}[1]{\textcolor[rgb]{0.81,0.36,0.00}{\textbf{#1}}}
\newcommand{\OtherTok}[1]{\textcolor[rgb]{0.56,0.35,0.01}{#1}}
\newcommand{\PreprocessorTok}[1]{\textcolor[rgb]{0.56,0.35,0.01}{\textit{#1}}}
\newcommand{\RegionMarkerTok}[1]{#1}
\newcommand{\SpecialCharTok}[1]{\textcolor[rgb]{0.00,0.00,0.00}{#1}}
\newcommand{\SpecialStringTok}[1]{\textcolor[rgb]{0.31,0.60,0.02}{#1}}
\newcommand{\StringTok}[1]{\textcolor[rgb]{0.31,0.60,0.02}{#1}}
\newcommand{\VariableTok}[1]{\textcolor[rgb]{0.00,0.00,0.00}{#1}}
\newcommand{\VerbatimStringTok}[1]{\textcolor[rgb]{0.31,0.60,0.02}{#1}}
\newcommand{\WarningTok}[1]{\textcolor[rgb]{0.56,0.35,0.01}{\textbf{\textit{#1}}}}
\usepackage{longtable,booktabs}
% Fix footnotes in tables (requires footnote package)
\IfFileExists{footnote.sty}{\usepackage{footnote}\makesavenoteenv{longtable}}{}
\usepackage{graphicx,grffile}
\makeatletter
\def\maxwidth{\ifdim\Gin@nat@width>\linewidth\linewidth\else\Gin@nat@width\fi}
\def\maxheight{\ifdim\Gin@nat@height>\textheight\textheight\else\Gin@nat@height\fi}
\makeatother
% Scale images if necessary, so that they will not overflow the page
% margins by default, and it is still possible to overwrite the defaults
% using explicit options in \includegraphics[width, height, ...]{}
\setkeys{Gin}{width=\maxwidth,height=\maxheight,keepaspectratio}
\setlength{\emergencystretch}{3em}  % prevent overfull lines
\providecommand{\tightlist}{%
  \setlength{\itemsep}{0pt}\setlength{\parskip}{0pt}}
\setcounter{secnumdepth}{5}
% Redefines (sub)paragraphs to behave more like sections
\ifx\paragraph\undefined\else
\let\oldparagraph\paragraph
\renewcommand{\paragraph}[1]{\oldparagraph{#1}\mbox{}}
\fi
\ifx\subparagraph\undefined\else
\let\oldsubparagraph\subparagraph
\renewcommand{\subparagraph}[1]{\oldsubparagraph{#1}\mbox{}}
\fi

% set default figure placement to htbp
\makeatletter
\def\fps@figure{htbp}
\makeatother

\usepackage{booktabs}
\usepackage{amsthm}
\makeatletter
\def\thm@space@setup{%
  \thm@preskip=8pt plus 2pt minus 4pt
  \thm@postskip=\thm@preskip
}
\makeatother

\usepackage{zxjatype}
\setmainfont[BoldFont = Noto Serif CJK JP]{Noto Serif CJK JP Light}
\setsansfont[BoldFont = Noto Sans CJK JP]{Noto Sans CJK JP Light}
\setmonofont{Noto Sans Mono CJK JP}
\setCJKmainfont[BoldFont = Noto Sans CJK JP]{Noto Sans CJK JP Light}
\setCJKsansfont[BoldFont = Noto Sans CJK JP]{Noto Sans CJK JP Light}

\usepackage{xcolor}
\definecolor{main}{HTML}{53727d}

\usepackage{hyperref}
\hypersetup{
  colorlinks = true,
  allcolors = main,
  pdfauthor = {},
  pdftitle = {},
  pdfsubject = {},
  pdfkeywords = {}
}
\usepackage[]{natbib}
\bibliographystyle{apalike}

\title{Rで計量政治学入門}
\author{土井 翔平}
\date{2020-04-18}

\begin{document}
\maketitle

{
\setcounter{tocdepth}{2}
\tableofcontents
}
\hypertarget{index}{%
\section*{はじめに}\label{index}}
\addcontentsline{toc}{section}{はじめに}

本書はRによる計量政治学の入門レベルの講義資料です。
質問や間違いなどがありましたら、\href{mailto:shohei.doi0504@gmail.com}{ご連絡}を下さい。
筆者のプロフィールは\href{https://shohei-doi.github.io/}{こちら}をご覧ください。

RやRStudioのインストールについては\protect\hyperlink{install-r}{Rの環境構築}を、基本操作については\protect\hyperlink{intro-r}{Rプログラミング入門}をご覧ください。

\hypertarget{part-ux30c7ux30fcux30bfux30cfux30f3ux30c9ux30eaux30f3ux30b0}{%
\part{データ・ハンドリング}\label{part-ux30c7ux30fcux30bfux30cfux30f3ux30c9ux30eaux30f3ux30b0}}

\hypertarget{data-import}{%
\section{データの読み込み}\label{data-import}}

本章ではデータを読み込む方法について解説します。

\hypertarget{ux4f7fux7528ux3059ux308bux30d1ux30c3ux30b1ux30fcux30b8}{%
\subsection{使用するパッケージ}\label{ux4f7fux7528ux3059ux308bux30d1ux30c3ux30b1ux30fcux30b8}}

\begin{Shaded}
\begin{Highlighting}[]
\KeywordTok{library}\NormalTok{(tidyverse)}
\end{Highlighting}
\end{Shaded}

\hypertarget{appendix-ux88dcux8ad6}{%
\appendix}


\hypertarget{install-r}{%
\section{Rの環境構築}\label{install-r}}

\hypertarget{intro-r}{%
\section{Rプログラミング入門}\label{intro-r}}

\hypertarget{ux95a2ux6570}{%
\subsection{関数}\label{ux95a2ux6570}}

\textbf{関数} (function) とは何かを入力すると、何かを出力するものです。
例えば、

\begin{Shaded}
\begin{Highlighting}[]
\KeywordTok{print}\NormalTok{(}\StringTok{"Hello, World."}\NormalTok{)}
\end{Highlighting}
\end{Shaded}

\begin{verbatim}
## [1] "Hello, World."
\end{verbatim}

というコードは、\texttt{"Hello,\ World."}という文字列を\texttt{print()}という関数に入力し、その文字列を出力しています。

\begin{itemize}
\tightlist
\item
  Rでは、関数は\texttt{関数名()}という形を取ります。
\item
  入力するものを入力引数 (input argument) 、出力するものを出力引数 (output argument) と呼んだりします。
\end{itemize}

次のように、入力引数も出力引数も1つとは限りません。

\begin{Shaded}
\begin{Highlighting}[]
\KeywordTok{rnorm}\NormalTok{(}\DataTypeTok{n =} \DecValTok{10}\NormalTok{, }\DataTypeTok{mean =} \DecValTok{0}\NormalTok{, }\DataTypeTok{sd =} \DecValTok{1}\NormalTok{)}
\end{Highlighting}
\end{Shaded}

\begin{verbatim}
##  [1]  0.5100836 -1.9178627  0.3617162  1.6663065  0.6966553  0.9176074
##  [7] -0.9652590 -1.8262058  1.3781373  0.3726935
\end{verbatim}

さて、この関数は何をしているのでしょうか。
Rでは、関数名の前に\texttt{?}をつけて実行することで、その関数のヘルプを見ることができます。

\begin{Shaded}
\begin{Highlighting}[]
\NormalTok{?rnorm}
\end{Highlighting}
\end{Shaded}

英語で関数の使い方が解説されていますが、\texttt{rnorm(n\ =\ 10,\ mean\ =\ 0,\ sd\ =\ 1)}は平均0、標準偏差1の(標準)正規分布に従う乱数を10個だけ生じさせています。

入力引数は\texttt{=}で明示的に指定する場合、どのような順番でも構いません。

\begin{Shaded}
\begin{Highlighting}[]
\KeywordTok{rnorm}\NormalTok{(}\DataTypeTok{mean =} \DecValTok{0}\NormalTok{, }\DataTypeTok{sd =} \DecValTok{1}\NormalTok{, }\DataTypeTok{n =} \DecValTok{10}\NormalTok{)}
\end{Highlighting}
\end{Shaded}

入力引数を明示的に指定しない場合、ヘルプにある順番で入力します。
以下の例は上述のものと同じです。

\begin{Shaded}
\begin{Highlighting}[]
\KeywordTok{rnorm}\NormalTok{(}\DecValTok{10}\NormalTok{, }\DecValTok{0}\NormalTok{, }\DecValTok{1}\NormalTok{)}
\end{Highlighting}
\end{Shaded}

また、ヘルプで\texttt{mean\ =\ 0,\ sd\ =\ 1}のように書かれている場合、デフォルトが定められています。
実行者が入力引数を指定しない限り、デフォルト値が使用されます。
したがって、以下の例もこれまでと同じコードです。
x

\begin{Shaded}
\begin{Highlighting}[]
\KeywordTok{rnorm}\NormalTok{(}\DecValTok{10}\NormalTok{)}
\end{Highlighting}
\end{Shaded}

\hypertarget{ux30aaux30d6ux30b8ux30a7ux30afux30c8}{%
\subsection{オブジェクト}\label{ux30aaux30d6ux30b8ux30a7ux30afux30c8}}

Rでは\texttt{\textless{}-}でオブジェクトを作成することができます。
例えば、100個の正規分布に従う乱数を\texttt{x}という名前のオブジェクトとして作成します。

\begin{Shaded}
\begin{Highlighting}[]
\NormalTok{x <-}\StringTok{ }\KeywordTok{rnorm}\NormalTok{(}\DecValTok{100}\NormalTok{)}
\end{Highlighting}
\end{Shaded}

\begin{itemize}
\tightlist
\item
  RStudioでは\texttt{\textless{}-}はショートカット\texttt{Alt\ +\ -}で入力できます。
\end{itemize}

\begin{Shaded}
\begin{Highlighting}[]
\NormalTok{x}
\end{Highlighting}
\end{Shaded}

\begin{verbatim}
##   [1] -0.634329890 -0.933581635  0.500597309 -0.237829713  1.134620433
##   [6]  0.157317447 -1.278441403 -0.169100490  0.953810676 -0.653200473
##  [11] -0.796310222  0.221745769  1.113032644  0.851603629  0.389395880
##  [16] -0.060482461  0.532422515  0.130012414  0.916426646 -1.130660001
##  [21]  1.519160421 -2.094265007  1.478230782  0.029008353 -1.062132403
##  [26]  0.011818349 -0.153622769 -0.322972427 -1.214824810 -0.752159414
##  [31]  0.598106395 -0.223620657 -0.077004346  1.051250545  0.017722334
##  [36]  0.331179280  0.368518002 -0.045472980  0.867805121 -0.767005798
##  [41]  0.154741473  0.004048953 -1.463517484  1.780155568 -0.626764309
##  [46] -0.989029395 -0.519081426  1.311618083  1.710519721  0.675993705
##  [51] -1.629270254  0.158816347 -1.152727667  2.876251126  0.303803216
##  [56] -0.020521977  0.542121676  0.529408201  0.393019694 -1.206772558
##  [61] -0.563868534 -1.578226881 -0.105160658 -0.066183955  1.165629714
##  [66] -2.146351536 -1.535515668  0.723146042  2.674476389  1.049104757
##  [71] -1.119754490 -1.001463143  0.989195459  0.356253324  0.568064365
##  [76] -1.683334863 -1.155942467 -0.857308916  1.127182373 -0.964649493
##  [81]  0.235484780 -1.904939252  0.492043196  0.711809763  0.026034740
##  [86] -0.137748542 -0.267288545 -0.373703486  0.980628346  0.067536281
##  [91] -0.112648097 -2.473260415 -0.218552691  0.027380917 -0.438420060
##  [96] -0.579576355 -0.878651361 -0.912948960  1.264377248 -1.464715491
\end{verbatim}

実際に、乱数が\texttt{x}に格納されていることが分かります。

オブジェクトを入力引数とすることも可能です。
\texttt{x}の平均と標準偏差を求めてみます。

\begin{Shaded}
\begin{Highlighting}[]
\KeywordTok{mean}\NormalTok{(x)}
\end{Highlighting}
\end{Shaded}

\begin{verbatim}
## [1] -0.06682285
\end{verbatim}

\begin{Shaded}
\begin{Highlighting}[]
\KeywordTok{sd}\NormalTok{(x)}
\end{Highlighting}
\end{Shaded}

\begin{verbatim}
## [1] 1.010005
\end{verbatim}

もちろん、出力引数を新しいオブジェクトにすることもできます。

\begin{Shaded}
\begin{Highlighting}[]
\NormalTok{x.mean <-}\StringTok{ }\KeywordTok{mean}\NormalTok{(x)}
\NormalTok{x.mean}
\end{Highlighting}
\end{Shaded}

\begin{verbatim}
## [1] -0.06682285
\end{verbatim}

\begin{itemize}
\tightlist
\item
  オブジェクトの名前にはアルファベットと数字、\texttt{.}と\texttt{\_}が使えます。
\item
  ただし、数字は最初の文字としては使えません。
\end{itemize}

オブジェクトは上書きすることもできます。

\begin{Shaded}
\begin{Highlighting}[]
\NormalTok{x.mean <-}\StringTok{ }\KeywordTok{mean}\NormalTok{(}\KeywordTok{rnorm}\NormalTok{(}\DecValTok{100}\NormalTok{))}
\end{Highlighting}
\end{Shaded}

\hypertarget{ux30d1ux30c3ux30b1ux30fcux30b8}{%
\subsection{パッケージ}\label{ux30d1ux30c3ux30b1ux30fcux30b8}}

大雑把に言って、Rによるデータ分析は\textbf{データをオブジェクトとして読み込み、いろいろな関数で処理を行うこと}で実行します。

つまり、関数が重要なのですが、Rで標準に備わっている関数には限界があります。
そこで、様々な研究者が関数を作成し、それをまとめたものを\textbf{パッケージ}として公開しています。

\begin{itemize}
\tightlist
\item
  基本的に、\href{https://cran.r-project.org/}{CRAM}でパッケージは公開されます。
\end{itemize}

パッケージをインストールするには、\texttt{install.packages()}という関数にパッケージ名を入れて実行します。
試しに、\href{https://www.tidyverse.org/}{Tidyverse}という幅広く使われているパッケージをインストールしてみます。

\begin{Shaded}
\begin{Highlighting}[]
\KeywordTok{install.packages}\NormalTok{(}\StringTok{"tidyverse"}\NormalTok{)}
\end{Highlighting}
\end{Shaded}

\begin{itemize}
\tightlist
\item
  RStudioの場合、\texttt{Packages}パネル(デフォルトの場合は右下)の中に\texttt{Install}というボタンがあり、そこにパッケージ名を入力していインストールすることも可能です。
\end{itemize}

\texttt{"}でパッケージ名を囲まないとエラーになります。

\begin{Shaded}
\begin{Highlighting}[]
\KeywordTok{install.packages}\NormalTok{(tidyverse)}
\end{Highlighting}
\end{Shaded}

\begin{verbatim}
## Error in install.packages(tidyverse): object 'tidyverse' not found
\end{verbatim}

インストールしたパッケージに対して再び\texttt{install.packages()}を行うと、最新版にアップデートされます。

\begin{itemize}
\tightlist
\item
  RStudioの場合、\texttt{Packages}パネルに\texttt{Update}というボタンがあり、アップデートできるパッケージを自動検索してくれます。
\end{itemize}

パッケージはインストールしただけでは使用することはできず、\texttt{library()}で読み込む必要があります。

\begin{Shaded}
\begin{Highlighting}[]
\KeywordTok{library}\NormalTok{(tidyverse)}
\end{Highlighting}
\end{Shaded}

\begin{itemize}
\tightlist
\item
  この場合は\texttt{"}で囲む必要はありません。
\end{itemize}

RStudioであれば\texttt{Packages}パネルにインストール済みのパッケージ一覧があるので、パッケージ名をクリックすると含まれる関数一覧を見ることができます。

\begin{itemize}
\tightlist
\item
  同様のものは\href{https://cran.r-project.org/}{CRAM}でも\texttt{pdf}形式で見ることができます。
\item
  一部のソフトウェアは\href{https://www.jstatsoft.org/index}{Journal of Statistical Software}などで論文が公開されています。
\end{itemize}

\hypertarget{rux30b9ux30afux30eaux30d7ux30c8}{%
\subsection{Rスクリプト}\label{rux30b9ux30afux30eaux30d7ux30c8}}

\hypertarget{rux30d7ux30edux30b8ux30a7ux30afux30c8}{%
\subsection{Rプロジェクト}\label{rux30d7ux30edux30b8ux30a7ux30afux30c8}}

\hypertarget{ux95a2ux6570ux306eux4f5cux6210}{%
\subsection{関数の作成}\label{ux95a2ux6570ux306eux4f5cux6210}}

Rで関数を自作する際は\texttt{function()\{\}}という関数を使います。

\begin{itemize}
\tightlist
\item
  \texttt{()}の中に入力引数を記述します。
\item
  \texttt{\{\}}の中に処理内容を記述し、最後に\texttt{return()}で出力引数を指定します。
\end{itemize}

例えば、数値ベクトルを入力引数として、平均と標準偏差を出力引数とする関数を作成します。

\begin{Shaded}
\begin{Highlighting}[]
\NormalTok{mean_sd <-}\StringTok{ }\ControlFlowTok{function}\NormalTok{(x) \{ }\CommentTok{# 入力引数の名前をxとしておきます。}
\NormalTok{  mean.x <-}\StringTok{ }\KeywordTok{mean}\NormalTok{(x) }\CommentTok{# 平均を計算します。}
\NormalTok{  sd.x <-}\StringTok{ }\KeywordTok{sd}\NormalTok{(x) }\CommentTok{# 標準偏差を計算します。}
  \KeywordTok{return}\NormalTok{(}\KeywordTok{c}\NormalTok{(mean.x, sd.x)) }\CommentTok{# 出力引数を指定します。}
\NormalTok{\}}
\end{Highlighting}
\end{Shaded}

実際に実行してみます。

\begin{Shaded}
\begin{Highlighting}[]
\NormalTok{x <-}\StringTok{ }\KeywordTok{rnorm}\NormalTok{(}\DecValTok{100}\NormalTok{)}
\KeywordTok{mean_sd}\NormalTok{(x)}
\end{Highlighting}
\end{Shaded}

\begin{verbatim}
## [1] 0.04921945 1.03281027
\end{verbatim}

\hypertarget{ux30ebux30fcux30d7}{%
\subsection{ループ}\label{ux30ebux30fcux30d7}}

\textbf{ループ}とは同一の処理を複数回実行することを指します。
例えば、100個の標準正規分布に従う乱数の平均を5回求める処理は次のようになります。

\begin{Shaded}
\begin{Highlighting}[]
\ControlFlowTok{for}\NormalTok{ (i }\ControlFlowTok{in} \DecValTok{1}\OperatorTok{:}\DecValTok{5}\NormalTok{) \{}
  \KeywordTok{print}\NormalTok{(}\KeywordTok{mean}\NormalTok{(}\KeywordTok{rnorm}\NormalTok{(}\DecValTok{100}\NormalTok{)))}
\NormalTok{\}}
\end{Highlighting}
\end{Shaded}

\begin{verbatim}
## [1] 0.06185255
## [1] -0.0497178
## [1] -0.02262662
## [1] 0.1639038
## [1] -0.07830911
\end{verbatim}

\texttt{for}ループとは\texttt{()}の中の\texttt{in}のあとのベクトルの第1要素から順番に\texttt{i}に代入して繰り返しています。
そのことは、次の例から解ると思います。

\begin{Shaded}
\begin{Highlighting}[]
\KeywordTok{head}\NormalTok{(letters)}
\end{Highlighting}
\end{Shaded}

\begin{verbatim}
## [1] "a" "b" "c" "d" "e" "f"
\end{verbatim}

\begin{itemize}
\tightlist
\item
  \texttt{letters}とはアルファベットのベクトルです。
\end{itemize}

\begin{Shaded}
\begin{Highlighting}[]
\ControlFlowTok{for}\NormalTok{ (i }\ControlFlowTok{in} \KeywordTok{head}\NormalTok{(letters)) \{}
  \KeywordTok{print}\NormalTok{(i)}
\NormalTok{\}}
\end{Highlighting}
\end{Shaded}

\begin{verbatim}
## [1] "a"
## [1] "b"
## [1] "c"
## [1] "d"
## [1] "e"
## [1] "f"
\end{verbatim}

\begin{itemize}
\tightlist
\item
  \texttt{for}ループとは別に、特定の条件が満たされるまで繰り返される\texttt{while}ループもあります。
\end{itemize}

ループ処理の結果を格納するには少しテクニックが必要です。
100個の乱数の平均を5回取ったものを\texttt{x}として保存したいとします。

まず、\texttt{x}を\texttt{NULL}オブジェクトとして作成します。

\begin{Shaded}
\begin{Highlighting}[]
\NormalTok{x <-}\StringTok{ }\OtherTok{NULL}
\NormalTok{x}
\end{Highlighting}
\end{Shaded}

\begin{verbatim}
## NULL
\end{verbatim}

\begin{itemize}
\tightlist
\item
  \texttt{NULL}とは空っぽのオブジェクト(0という数値や空白という文字ではない)です。
\end{itemize}

先程のループ処理の中で、計算した平均を\texttt{c()}で\texttt{x}にくっつけていきます。

\begin{Shaded}
\begin{Highlighting}[]
\ControlFlowTok{for}\NormalTok{ (i }\ControlFlowTok{in} \DecValTok{1}\OperatorTok{:}\DecValTok{5}\NormalTok{) \{}
\NormalTok{  x <-}\StringTok{ }\KeywordTok{c}\NormalTok{(x, }\KeywordTok{mean}\NormalTok{(}\KeywordTok{rnorm}\NormalTok{(}\DecValTok{100}\NormalTok{)))}
\NormalTok{\}}
\NormalTok{x}
\end{Highlighting}
\end{Shaded}

\begin{verbatim}
## [1]  0.04482331  0.15245047 -0.20079446  0.14338999  0.09254384
\end{verbatim}

無事、5個の平均値が\texttt{x}に保存されていることがわかります。

実際に\texttt{for}ループの中で何が起こっているかは、次のコードで解ると思います。

\begin{Shaded}
\begin{Highlighting}[]
\NormalTok{x <-}\StringTok{ }\OtherTok{NULL}
\ControlFlowTok{for}\NormalTok{ (i }\ControlFlowTok{in} \DecValTok{1}\OperatorTok{:}\DecValTok{5}\NormalTok{) \{}
\NormalTok{  x <-}\StringTok{ }\KeywordTok{c}\NormalTok{(x, }\KeywordTok{mean}\NormalTok{(}\KeywordTok{rnorm}\NormalTok{(}\DecValTok{100}\NormalTok{)))}
  \KeywordTok{print}\NormalTok{(x)}
\NormalTok{\}}
\end{Highlighting}
\end{Shaded}

\begin{verbatim}
## [1] 0.05292771
## [1] 0.05292771 0.06922472
## [1]  0.05292771  0.06922472 -0.01338469
## [1]  0.05292771  0.06922472 -0.01338469  0.24274790
## [1]  0.05292771  0.06922472 -0.01338469  0.24274790  0.20090668
\end{verbatim}

\begin{itemize}
\tightlist
\item
  ループが一周するたびに、前回の\texttt{x}に新しい要素が付け加わり、新しい\texttt{x}として保存されています。
\end{itemize}

\texttt{NULL}オブジェクトを使ったループ結果の保存でよくあるミスは、やり直す際に\texttt{NULL}でリセットするのを忘れることです。
例えば、同じコードをもう一度実行しましょう。

\begin{Shaded}
\begin{Highlighting}[]
\ControlFlowTok{for}\NormalTok{ (i }\ControlFlowTok{in} \DecValTok{1}\OperatorTok{:}\DecValTok{5}\NormalTok{) \{}
\NormalTok{  x <-}\StringTok{ }\KeywordTok{c}\NormalTok{(x, }\KeywordTok{mean}\NormalTok{(}\KeywordTok{rnorm}\NormalTok{(}\DecValTok{100}\NormalTok{)))}
\NormalTok{\}}
\NormalTok{x}
\end{Highlighting}
\end{Shaded}

\begin{verbatim}
##  [1]  0.05292771  0.06922472 -0.01338469  0.24274790  0.20090668  0.06541486
##  [7]  0.14932941 -0.08364625  0.03719465  0.07494928
\end{verbatim}

\begin{itemize}
\tightlist
\item
  \texttt{x}に10個の平均値が入っています。
\end{itemize}

このようなミスを避ける方法の一つは、全体を関数として作成することです。

\begin{Shaded}
\begin{Highlighting}[]
\NormalTok{multi_mean <-}\StringTok{ }\ControlFlowTok{function}\NormalTok{() \{}
\NormalTok{  x <-}\StringTok{ }\OtherTok{NULL}
  \ControlFlowTok{for}\NormalTok{ (i }\ControlFlowTok{in} \DecValTok{1}\OperatorTok{:}\DecValTok{5}\NormalTok{) \{}
\NormalTok{    x <-}\StringTok{ }\KeywordTok{c}\NormalTok{(x, }\KeywordTok{mean}\NormalTok{(}\KeywordTok{rnorm}\NormalTok{(}\DecValTok{100}\NormalTok{)))}
\NormalTok{  \}}
  \KeywordTok{return}\NormalTok{(x)}
\NormalTok{\}}
\NormalTok{x <-}\StringTok{ }\KeywordTok{multi_mean}\NormalTok{()}
\NormalTok{x}
\end{Highlighting}
\end{Shaded}

\begin{verbatim}
## [1]  0.36437285  0.08879437 -0.07453318 -0.08601525  0.03688836
\end{verbatim}

\hypertarget{ux6761ux4ef6ux5206ux5c90}{%
\subsection{条件分岐}\label{ux6761ux4ef6ux5206ux5c90}}

\textbf{条件分岐}とは、特定の条件の場合に特定の動作を行うようにすることです。
例えば、正の場合\texttt{positive}、負の場合\texttt{negative}と出力するコマンドは次のようになります。

\begin{Shaded}
\begin{Highlighting}[]
\NormalTok{x <-}\StringTok{ }\KeywordTok{rnorm}\NormalTok{(}\DecValTok{1}\NormalTok{)}
\ControlFlowTok{if}\NormalTok{ (x }\OperatorTok{>}\StringTok{ }\DecValTok{0}\NormalTok{) \{}
  \KeywordTok{print}\NormalTok{(}\StringTok{"positive"}\NormalTok{)}
\NormalTok{\} }\ControlFlowTok{else}\NormalTok{ \{}
  \KeywordTok{print}\NormalTok{(}\StringTok{"negative"}\NormalTok{)}
\NormalTok{\}}
\end{Highlighting}
\end{Shaded}

\begin{verbatim}
## [1] "positive"
\end{verbatim}

\begin{Shaded}
\begin{Highlighting}[]
\KeywordTok{print}\NormalTok{(x)}
\end{Highlighting}
\end{Shaded}

\begin{verbatim}
## [1] 0.2608707
\end{verbatim}

\begin{itemize}
\tightlist
\item
  \texttt{if()\{\}}の\texttt{()}の中に条件式を書き、\texttt{\{\}}の中に処理内容を書きます。
\item
  それ以外の条件は\texttt{else}で示します。
\end{itemize}

条件式は3つ以上でも構いません。

\begin{Shaded}
\begin{Highlighting}[]
\NormalTok{x <-}\StringTok{ }\KeywordTok{rnorm}\NormalTok{(}\DecValTok{1}\NormalTok{)}
\ControlFlowTok{if}\NormalTok{ (x }\OperatorTok{>}\StringTok{ }\FloatTok{-0.5}\NormalTok{) \{}
  \KeywordTok{print}\NormalTok{(}\StringTok{"x is less than -0.5."}\NormalTok{)}
\NormalTok{\} }\ControlFlowTok{else} \ControlFlowTok{if}\NormalTok{ (x }\OperatorTok{>=}\StringTok{ }\FloatTok{-0.5} \OperatorTok{&}\StringTok{ }\NormalTok{x }\OperatorTok{<=}\StringTok{ }\FloatTok{0.5}\NormalTok{) \{}
  \KeywordTok{print}\NormalTok{(}\StringTok{"x is between -0.5 and 0.5."}\NormalTok{)}
\NormalTok{\} }\ControlFlowTok{else}\NormalTok{ \{}
  \KeywordTok{print}\NormalTok{(}\StringTok{"x is more than 0.5.ー"}\NormalTok{)}
\NormalTok{\}}
\end{Highlighting}
\end{Shaded}

\begin{verbatim}
## [1] "x is more than 0.5.ー"
\end{verbatim}

\begin{Shaded}
\begin{Highlighting}[]
\KeywordTok{print}\NormalTok{(x)}
\end{Highlighting}
\end{Shaded}

\begin{verbatim}
## [1] -2.039521
\end{verbatim}

\begin{itemize}
\tightlist
\item
  \texttt{\&}は「かつ」を意味します。
\item
  「または」は\texttt{\textbar{}}を使います。
\item
  \texttt{\textgreater{}=}は \(\geq\) を意味します。
\item
  「同じ値である」は\texttt{==}を使います(\texttt{=}ではない点に注意)。
\end{itemize}

\hypertarget{ux7df4ux7fd2ux554fux984cux30d5ux30a3ux30dcux30caux30c3ux30c1ux6570ux5217}{%
\subsection{練習問題:フィボナッチ数列}\label{ux7df4ux7fd2ux554fux984cux30d5ux30a3ux30dcux30caux30c3ux30c1ux6570ux5217}}

フィボナッチ数列とは以下の条件を満たす数列です。

\[
\begin{align}
  F_0 &= 0 \\
  F_1 &= 1 \\
  F_{n} &= F_{n-1} + F_{n-2} \quad n \geq 2
\end{align}
\]

例えば、

\[
\begin{align}
  F_2 = 1, F_3 = 2, F_4 = 3, F_5 = 5, F_6 = 8,\ldotsF
\end{align}
\]

となります。

フィボナッチ数列の第\(n\)項を(解析解を使わずに)求める関数を作成してみて下さい。

\hypertarget{ux7df4ux7fd2ux554fux984cux30e2ux30f3ux30c6ux30abux30ebux30edux30b7ux30dfux30e5ux30ecux30fcux30b7ux30e7ux30f3}{%
\subsection{練習問題:モンテカルロ・シミュレーション}\label{ux7df4ux7fd2ux554fux984cux30e2ux30f3ux30c6ux30abux30ebux30edux30b7ux30dfux30e5ux30ecux30fcux30b7ux30e7ux30f3}}

モンテカルロ・シミュレーション(モンテカルロ法)とは乱数を用いて近似解を求める手法です。

例えば、円周率\(\pi\)の近似解は以下のように求めることができます。

\begin{enumerate}
\def\labelenumi{\arabic{enumi}.}
\tightlist
\item
  0以上1未満の一様分布から\(n\)個の乱数\(x_i\)と\(n\)個の乱数\(y_i\)を発生させます (\(i = 1,2,\ldots,n\)) 。
\item
  原点と\((x_i,y_i)\)の距離が1以下である回数を計算し\(n_1\)とします。
\item
  円周率の近似解として\(\hat{\pi} = 4 \times n_1/n\)を得ます。
\end{enumerate}

モンテカルロ・シミュレーションによる円周率の近似解を求める関数を作成してみて下さい。

また、モンテカルロ・シミュレーションによる円周率の近似解を\(m\)回求めて、その平均値や標準偏差が\(n\)によってどのように変化するか検討してみて下さい。

\hypertarget{environment}{%
\section*{動作環境}\label{environment}}
\addcontentsline{toc}{section}{動作環境}

\begin{Shaded}
\begin{Highlighting}[]
\KeywordTok{sessionInfo}\NormalTok{()}
\end{Highlighting}
\end{Shaded}

\begin{verbatim}
## R version 3.6.3 (2020-02-29)
## Platform: x86_64-pc-linux-gnu (64-bit)
## Running under: Ubuntu 18.04.4 LTS
## 
## Matrix products: default
## BLAS:   /usr/lib/x86_64-linux-gnu/blas/libblas.so.3.7.1
## LAPACK: /usr/lib/x86_64-linux-gnu/lapack/liblapack.so.3.7.1
## 
## locale:
##  [1] LC_CTYPE=en_US.UTF-8       LC_NUMERIC=C              
##  [3] LC_TIME=en_US.UTF-8        LC_COLLATE=en_US.UTF-8    
##  [5] LC_MONETARY=en_US.UTF-8    LC_MESSAGES=en_US.UTF-8   
##  [7] LC_PAPER=en_US.UTF-8       LC_NAME=C                 
##  [9] LC_ADDRESS=C               LC_TELEPHONE=C            
## [11] LC_MEASUREMENT=en_US.UTF-8 LC_IDENTIFICATION=C       
## 
## attached base packages:
## [1] stats     graphics  grDevices utils     datasets  methods   base     
## 
## other attached packages:
## [1] forcats_0.5.0   stringr_1.4.0   dplyr_0.8.5     purrr_0.3.3    
## [5] readr_1.3.1     tidyr_1.0.2     tibble_3.0.0    ggplot2_3.3.0  
## [9] tidyverse_1.3.0
## 
## loaded via a namespace (and not attached):
##  [1] tidyselect_1.0.0 xfun_0.13        haven_2.2.0      lattice_0.20-41 
##  [5] colorspace_1.4-1 vctrs_0.2.4      generics_0.0.2   htmltools_0.4.0 
##  [9] yaml_2.2.1       rlang_0.4.5      pillar_1.4.3     withr_2.1.2     
## [13] glue_1.4.0       DBI_1.1.0        dbplyr_1.4.2     modelr_0.1.6    
## [17] readxl_1.3.1     lifecycle_0.2.0  munsell_0.5.0    gtable_0.3.0    
## [21] cellranger_1.1.0 rvest_0.3.5      evaluate_0.14    knitr_1.28      
## [25] fansi_0.4.1      broom_0.5.5      Rcpp_1.0.4.6     backports_1.1.6 
## [29] scales_1.1.0     jsonlite_1.6.1   fs_1.4.1         hms_0.5.3       
## [33] digest_0.6.25    stringi_1.4.6    bookdown_0.18    grid_3.6.3      
## [37] cli_2.0.2        tools_3.6.3      magrittr_1.5     crayon_1.3.4    
## [41] pkgconfig_2.0.3  ellipsis_0.3.0   xml2_1.3.1       reprex_0.3.0    
## [45] lubridate_1.7.8  assertthat_0.2.1 rmarkdown_2.1    httr_1.4.1      
## [49] rstudioapi_0.11  R6_2.4.1         nlme_3.1-144     compiler_3.6.3
\end{verbatim}

\bibliography{book.bib,packages.bib}

\end{document}
